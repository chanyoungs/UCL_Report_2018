\chapter{Some Appendix}
    This is just a bare minimum to get started.  There is unlimited guidance on using latex, e.g. {\tt https://en.wikibooks.org/wiki/LaTeX}.   You are still responsible to check the detailed requirements of a project, including formatting instructions, see \\
    {\tt https://moodle.ucl.ac.uk/pluginfile.php/3591429/mod\_resource/content/7/UGProjects2017.pdf}.
    Leave at least a line of white space when you want to start a new paragraph.
    
    Mathematical expressions are placed inline between dollar signs, e.g. $\sqrt 2, \sum_{i=0}^nf(i)$, or in display mode
    \[ e^{i\pi}=-1\] and another way, this time with labels,
    \begin{align}
    \label{line1} A=B\wedge B=C&\rightarrow A=C\\
    &\rightarrow C=A\\
    \intertext{note that}
    n!&=\prod_{1\leq i\leq n}i \\
    \int_{x=1}^y \frac 1 x \mathrm{d}x&=\log y
    \end{align}
    We can refer to labels like this \eqref{line1}.
    
    Often lots of citations here (and elsewhere), e.g. Bibtex can help with this, but is not essential. If you want pictures, try

    You can use 
    \begin{itemize}
    \item lists
    \item like this
    \end{itemize}
    or numbered
    \begin{enumerate}
    \item like this,
    \item or this
    \end{enumerate}
    but don't overdo it.
    
    If you have a formal theorem you might try this.
    \begin{definition}\label{def}
    See definition~\ref{def}.
    \end{definition}
    \begin{theorem}
    For all $n\in\nats,\; 1^n=1$.
    \end{theorem}
    \begin{proof}
    By induction over $n$.
    \end{proof}
